\documentclass[12pt]{article}
\usepackage[utf8]{inputenc}
\usepackage{amsmath}
\usepackage[slovene]{babel}
\usepackage{hyperref}
\title{Naloga 5}
\author{Gregor Kikelj}

\usepackage{natbib}
\usepackage{graphicx}
\usepackage{listings}
\usepackage{breqn}

\begin{document}
\section{Naloga 5}
Imamo projektivno krivuljo podano z $F=- y^{4} - z^{4} + 2 \left(x^{2} - 2 x y + y^{2} - y z\right)^{2}$. Izračunamo 
\[F_x = 2 \cdot \left(4 x - 4 y\right) \left(x^{2} - 2 x y + y^{2} - y z\right)=0\]
\[F_y = - 4 y^{3} + 2 \left(- 4 x + 4 y - 2 z\right) \left(x^{2} - 2 x y + y^{2} - y z\right)=0\]
\[F_z = - 4 y \left(x^{2} - 2 x y + y^{2} - y z\right) - 4 z^{3}=0\]
\bigskip

Enačbe delimo z 8 ali 4 da jih malo poenostavimo
\[{\left(x - y\right) \left(x^{2} - 2 x y + y^{2} - y z\right)}=0\]
\[- y^{3} - \left(2 x - 2 y + z\right) \left(x^{2} - 2 x y + y^{2} - y z\right)=0\]
\[- y \left(x^{2} - 2 x y + y^{2} - y z\right) - z^{3}=0\]
\bigskip

Iz prve enačbe delimo na 2 primera.
\bigskip

1.) $x-y=0$.
Sledi da je $y=x$ kar vstavimo v drugo in tretjo enačbo in poenostavimo da dobimo
\[- x^{3} + x z^{2}=0\]
\[x^{2} z - z^{3}=0\]
Če je $x=0$ potem dobimo $z=0$ kar ni v redu ker smo v projektivnem prostoru. Zato lahko delimo z $x$ da dobimo $-x^2+z^2=0$ oziroma $z^2=x^2$.
Delimo na 2 primera.
\bigskip

1.1) $z=x$.
Vidimo da to ustreza vsem enačbam kar nam da singularno točko $(1, 1, 1)$. 
\bigskip

1.2) $z=-x$.
To spet ustreza vsem enačbam in dobimo točko $(1, 1, -1)$.
\bigskip

2.) $x^{2} - 2 x y + y^{2} - y z=0$.
Tukaj imamo srečo in izraz prepoznamo v drugi in tretji enačbi ki se poenostavita da dobimo
\[-y^3=0\]
\[-z^3=0\]

Hitro vidimo da bo le trojica $(x, y, z)=(0, 0, 0)$ rešila ta sistem, zato ne dobimo singularnih točk v projektivnem.

\subsection{Preverjanje singularnih točk}
Sedaj za točki $(1, 1, 1)$ in $(1, 1, -1)$ preverimo, če res ležita na $C$. Če ju vstavimo v $F$ res dobimo $0$ tako da moramo le še preveriti rede.

Vidimo da je $F_{xx}(1,1,1)=8 x^{2} - 16 x y + 8 y^{2} - 8 y z + \left(2 x - 2 y\right) \left(8 x - 8 y\right)=-8\ne 0$, zato je točka reda 2.

Vidimo tudi da je $F_{xx}(1,1,-1)=8\ne 0$ po podobnem izračunu zato je ta točka tudi reda 2.

% Sedaj pa še tangente. Naj bo $x=1$ za projekcijo na afin prostor. Delamo s točko $(1,1,1)$ zato naj bo $t=y+1$ in $w=z+1$. Dobimo 
% $ y^{4} - z^{4} + 2 \left(y^{2} - y z - 2 y + 1\right)^{2}=0$ ampak nas zanimajo le najmanjši monomi. 
\end{document}
