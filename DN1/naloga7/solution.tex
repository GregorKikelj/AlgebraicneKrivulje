\documentclass[12pt]{article}
\usepackage[utf8]{inputenc}
\usepackage{amsmath}
\usepackage[slovene]{babel}
\usepackage{hyperref}
\title{Naloga 7}
\author{Gregor Kikelj}

\usepackage{natbib}
\usepackage{graphicx}
\usepackage{listings}
\usepackage{breqn}

\begin{document}
\section{Naloga 7}
Vemo da se premice v projektivni ravnini vedno sekajo. Naj bo $A_1=L1\cap L2$, $A_2=L2\cap L3$, $A_3=L3\cap L4$, $A_4=L4\cap L1$,
$B_1=M_1\cap M_2$, $B_2=M_2\cap M_3$, $B_3=M_3\cap M_4$ in $B_4=M_4\cap M_1$. Naj bo $P$ projektivnost ki slika $A_i$ v $B_i$ za vse $i\in \{1, 2, 3, 4\}$.
To obstaja po lemi iz predavanj. BŠS bomo dokazali le da je $P(A_1)=M_1$, za ostale premice je dokaz enak. Ker se nobene tri premice ne sekajo v isti tocki je 
$A_1\neq A_4$. Torej točki $A_1$ in $A_4$ definirata premico $L1$. Podobno tudi $B_1$ in $B_4$ definirata premico $M_1$. Ker $P$ slika $A_1$ v $B_1$ in $A_2$ v $B_2$
potem tudi slika $L_1$ v $M_1$ saj projektivnosti slikajo premice v premice.

\end{document}
