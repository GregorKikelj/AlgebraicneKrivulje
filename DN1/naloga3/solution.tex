\documentclass[12pt]{article}
\usepackage[utf8]{inputenc}
\usepackage{amsmath}
\usepackage[slovene]{babel}
\usepackage{hyperref}
\title{Naloga 3}
\author{Gregor Kikelj}

\usepackage{natbib}
\usepackage{graphicx}
\usepackage{listings}


\begin{document}
\section{Naloga 3}
Imamo izraz $F(x, y)=(x^2+y^2)^3-4x^2y^2$. Dobimo \[F_x=- 8 x y^{2} + 6 x \left(x^{2} + y^{2}\right)^{2}=2 x \left(3 x^{4} + 6 x^{2} y^{2} + 3 y^{4} - 4 y^{2}\right)\]
in \[F_y=- 8 x^{2} y + 6 y \left(x^{2} + y^{2}\right)^{2}=2 y \left(3 x^{4} + 6 x^{2} y^{2} - 4 x^{2} + 3 y^{4}\right)\]

Precej očitno je da je $(0, 0)$ singularna točka, saj je tudi $F(0, 0)=0$. Dokazali bomo da je to edina singularna točka.
Precej očitno je tudi da če je $x=0$ je tudi $y=0$ in obratno. Zato se omejimo na primere 
$x\neq 0$ in $y\neq 0$.

Rešujemo sistem $F_x=0$ in $F_y=0$ kar se s predpostavko $x\neq 0$ in $y\neq 0$ poenostavi v $3 x^{4} + 6 x^{2} y^{2} + 3 y^{4} - 4 y^{2}=0$ in $3 x^{4} + 6 x^{2} y^{2} - 4 x^{2} + 3 y^{4}=0$
Če od prve enačbe odštejemo drugo dobimo $4x^2-4y^2=0$, torej je $x^2=y^2$. Vstavimo v prvo enačbo in dobimo $3x^4+6x^4+3x^4-4x^2=0$ kar preprosto rešimo da dobimo $x=\pm \sqrt{3}$ in enako $y=\pm \sqrt{3}$. 
S preprostim preizkusom vidimo da te točke niso na krivulji zato niso singularne točke.

Red točke $(0, 0)$ je očitno $4$ ker je to stopnja najmanjšega monoma. 

Sedaj pridobimo tangente kot rešitve $F=0$ kjer pa vzamemo le monome stopnje $4$. Dobimo $-4x^2y^2=0$ od koder dobimo tangenti $x=0$ in $y=0$. Določimo presečne večkratnosti obeh tangent.

Če v $F$ vstavimo $x=0$ dobimo $-y^6=0$, torej je presečna večkratnost enaka 6. 

Če v $F$ vstavimo $y=0$ dobimo $x^6=0$, torej je presečna večkratnost spet enaka 6. 


\end{document}
