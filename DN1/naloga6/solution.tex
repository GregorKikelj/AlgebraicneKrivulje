\documentclass[12pt]{article}
\usepackage[utf8]{inputenc}
\usepackage{amsmath}
\usepackage[slovene]{babel}
\usepackage{hyperref}
\title{Naloga 6}
\author{Gregor Kikelj}

\usepackage{natbib}
\usepackage{graphicx}
\usepackage{listings}
\usepackage{breqn}

\begin{document}
\section{Naloga 6}
Naj bo $q=x^{4} + y^{3} + 4 y^{2} + 6 y + 3$. Uporabimo eisenstienov kriterij zato pišemo $q=(y^3+4y^2+6y+3)x^0+x^4$. Ugibamo $p=y+1$. $a_0\in (p)$ ker je 
$y^3+4y^2+6y+3=(y+1)(y^2+3y+3)$ S kvadratno formulo se hitro preveri da $y^2+3y+3$ ni razcepen zato $a_0$ ni element $(p^2)$. $a_4=1$ tudi očitno ni element $(p)$ s tem pa smo preverili
vse pogoje.

Maksimalni ideali pa so vsi ideali oblike $(x-a, y-b)$ kjer je $(a, b)$ element krivulje $p=0$, na primer $(x, y+1)$.

\end{document}
