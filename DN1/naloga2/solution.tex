\documentclass[12pt]{article}
\usepackage[utf8]{inputenc}
\usepackage{amsmath}
\usepackage[slovene]{babel}
\usepackage{hyperref}
\title{Naloga 2}
\author{Gregor Kikelj}

\usepackage{natbib}
\usepackage{graphicx}
\usepackage{listings}


\begin{document}
\section{Naloga 2}
S homogenizacijo dobimo $F=x^3-x^2y-5y^2z-2xyz-xz^2$ in $G=x^2-xy+xz+yz$. Vidimo da $[1, 0, 0]$ ne leži v $F\cap G$ zato računamo rezultanto po $x$ koordinati. 
Dobimo matriko \[\left[\begin{matrix}1 & - y & - 2 y z - z^{2} & - 5 y^{2} z & 0\\0 & 1 & - y & - 2 y z - z^{2} & - 5 y^{2} z\\1 & - y + z & y z & 0 & 0\\0 & 1 & - y + z & y z & 0\\0 & 0 & 1 & - y + z & y z\end{matrix}\right]\],
ki nam da rezultanto $45 y^{4} z^{2} - 18 y^{3} z^{3} + 5 y^{2} z^{4}$ 

To se razstavi v
$45 y^{2} z^{2} \left(y + z \left(- \frac{1}{5} - \frac{4 i}{15}\right)\right) \left(y + z \left(- \frac{1}{5} + \frac{4 i}{15}\right)\right)$.

Delamo po delih.


$y=0$ nam da $F=x^3-xz^2$ in $G=x^2+xz$. Dobimo rešitvi $[0, 0, 1]$ in $[1, 0, -1]$. Vsota večkratnosti presečišč je 2, zato imata obe presečišči stopnjo $1$. 

$z=0$ nam da $F=x^3-x^2y=x^2(x-y)$ in $G=x^2-xy=x\cdot (x-y)$. Dobimo rešitvi $[0, 1, 0]$(pri primeru $x=0$) in $[1, 1, 0]$(pri primeru $x-y=0$). Vsota večkratnosti presečišč je 2, zato imata obe presečišči stopnjo $1$.

Če vstavimo $y=z(\frac{1}{5} +\frac{4i}{15} )$ dobimo \[F=x^{3} - x^{2} z \left(\frac{1}{5} + \frac{4 i}{15}\right) - x z^{2} - 2 x z^{2} \cdot \left(\frac{1}{5} + \frac{4 i}{15}\right) - 5 z^{3} \left(\frac{1}{5} + \frac{4 i}{15}\right)^{2}\]
in \[G=x^{2} + x z - x z \left(\frac{1}{5} + \frac{4 i}{15}\right) + z^{2} \cdot \left(\frac{1}{5} + \frac{4 i}{15}\right)\]
Presečišče bo tukaj večkratnosti $1$ in samo eno kar vidimo iz rezultante. $F$ in $G$ lahko razstavimo kot
\[F=\left(x + \frac{i z}{3}\right) \left(x^{2} + x z \left(- \frac{1}{5} - \frac{3 i}{5}\right) + z^{2} \left(- \frac{8}{5} - \frac{7 i}{15}\right)\right)\]
in
\[G=\left(x + \frac{i z}{3}\right) \left(x + z \left(\frac{4}{5} - \frac{3 i}{5}\right)\right)\]
Vidimo da je $x+\frac{iz}{3}$ skupni faktor in dobimo presečišče $[-i/3, 0, 1]$.

Podobno dobimo pri $y=z(\frac{1}{5} -\frac{4i}{15} )$ kjer je potem 
\[F=\left(x - \frac{i z}{3}\right) \left(x^{2} + x z \left(- \frac{1}{5} + \frac{3 i}{5}\right) + z^{2} \left(- \frac{8}{5} + \frac{7 i}{15}\right)\right)\]
in
\[G=\left(x - \frac{i z}{3}\right) \left(x + z \left(\frac{4}{5} + \frac{3 i}{5}\right)\right)\]
Spet je vsota večkratnosti ena torej le eno presečišče ki je $[-i/3, 0, 1]$.

S tem smo našteli vsa presečišča ki jih je 6 in imajo vsa večkratnost 1.
\end{document}
